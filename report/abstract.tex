\documentclass[./main.tex]{subfiles}

\begin{document}
\section*{Abstract}
In this thesis we implement four architectures for extending an already developed keypoint detector for bouldering. The three architectures consist of (1) a single 3-dimensional convolutional layer followed by the ReLU activation function, (2) DeciWatch by Zeng \textit{Et al.} \cite{https://doi.org/10.48550/arxiv.2203.08713}, and (3) two kinds of bidirectional convolutional LSTMs inspired by Unipise-LSTM by Artacho and Savakis \cite{https://doi.org/10.48550/arxiv.2001.08095}, where the difference between the two architectures lies in how they combine the two processing directions. The models are pretrained on the BRACE and Penn Action datasets and finetuned on a dataset for bouldering. The keypoint detector and the finetuning dataset are both provided by ClimbAlong at NorthTech ApS. We perform various experiments to find the optimal setting of the four models. Finally, we conclude, that DeciWatch by Zeng \textit{Et al.} \cite{https://doi.org/10.48550/arxiv.2203.08713} yielding the most accurate results, one of the bidirectional convolutional LSTMs yielded the best rough estimations, as well as the simple 3-dimensional convolutional layer yielded the best results when also considering the size and prediction time of the model.

\end{document}
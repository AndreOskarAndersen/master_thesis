\documentclass[./main.tex]{subfiles}

\begin{document}
\section{Conclusion}
\label{sec:conclusion}
Throughout this thesis we have succesfully developed and testign various machine learning models for extending an already developed keypoint detector for bouldering, such that it makes use of temporal smoothing. This was done by pretraining these models on the BRACE and Penn Action datasets and further finetune them on a dataset for bouldering, provded by ClimbAlong at Northtech ApS. Further, three experiments with two different setups were run for each model, such that we would end up with the optimal setting of each model. Lastly, we discussed our approach, including our results as well as mistakes we have made, as well as argued that the optimal choice of model depended on ones needs. Generally however, we find that DeciWatch by Zeng \textit{Et al.} \cite{https://doi.org/10.48550/arxiv.2203.08713} yielding the most accurate results, the bidirectional convolutional LSTM that uses a convolutional layer for combining the two processing directions yielded the best rough estimations, as well as the simple 3-dimensional convolutional layer yielded the best results when also considering the size and prediction time of the model.


\end{document}
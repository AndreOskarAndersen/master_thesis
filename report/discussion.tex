\documentclass[./main.tex]{subfiles}

\begin{document}
\section{Discussion}
The following section discusses our obtained results from the pretraining and finetuning stages, some concerns regarding possible suboptimal configuration choices we have made, as well as suggests some possible future work. 

\subsection{Results}
\label{sec:results}
In section \ref{sec:pretraining} and section \ref{sec:finetuning} we successfully implemented and tested the four architectures in various experiments. We will in the following be analysing and discussing the potential reasons for these obtained results.

\subsubsection{Pretraining}
We saw in section \ref{subsec:pretrain_train_val_res} how the 3-dimensional convolutional layer tend to perform much better in experiment 2, than it does in experiment 1 (of course with the exception of run 2.1 and 2.2). As the goal of experiment 1 is for the models to both learn translation and scaling, whereas the goal of experiment 2 is just for the models to learn translation, we can clearly see here how big of a difference the task of learning to also scale the data has on the results for the 3-dimensional convolutional layer. Of course, a tiny bit of randomness does also play in here, as the models for the two experiments were initialized with different weights. However, as the weights were sampled from the same distribution, as well as the 3-dimensional convolutional layer having very few weights, we find it hard to believe that this randomness has such a big impact on the results. Thus we claim that the performance differences between the two experiments is due to the task of scaling the data.
\\
\\
For the two architectures that are based on a bidirectional convolutional LSTM, a similar pattern is observed for the shifting scalar, however, this performance difference is much smaller than it was in the case for the 3-dimensional convolutional layer. We simply believe, that this improved performance is due to the bidirectional convolutional LSTMs predicting the position of the keypoints in a more sophisticated manner.
\\
\\
For DeciWatch there are no performance differences between experiment 1 and experiment 2 when considering shifting-scalar $s = 1$. However, this is expected, as unlike the other architectures, DeciWatch works directly on keypoint-coordinates instead of the heatmaps, making it scaling invariant and thus making it the two experiments equivalent to each other. On the other hand however, when considering shifting scalar $s = 2$, there is a clear performance difference between experiment 1 and experiment 2 for DeciWatch, which is very surprising for us, as the model is scaling invariant. However, if we inspect Figure \ref{fig:pretraining_res} we can clearly see, that the initial performances of DeciWatch 2.1 and DeciWatch 2.2 are much farther away from each other, than the initial performances of DeciWatch 1.1 and DeciWatch 1.2 were from each other. Thus, we suggest that the final performance difference between DeciWatch 2.1 and DeciWatch 2.2 is simply due to the randomness in the weight initalization of the two models.
\\
\\
We further saw in section \ref{subsec:pretrain_test_res} the effects on the models of halfing the frame rate. For both the 3-dimensional convolutional LSTM and DeciWatch this had a negative effect, whereas it actually had a positive effect for the two models that are based on a bidirectional LSTM. For DeciWatch we have to keep in mind, that it is already only considering every fifth frame, so by halfing the frame rate it is actually only considering every tenth frame, giving it a lot less context for it to process. For the 3-dimensional convolutional layer and the two models based on a bidirectional convolutional LSTM, we again suggest that the performance difference is due to the later models yielding their predictions in a more sophisticated manner than the simple 3-dimensional convolutional layer. In the case of when the shifting-scalar is $s = 2$ all models generally perform worse in experiment 3 than they do in experiment 1, which we simply believe is due to the data being so noisy, that the models need the finer details to infer their predictions. As the distance-threshold for PCK is increased however, the 3-dimensional convolutional layer actually perform better in experiment 3 than it does in experiment 1. However, we have to note here, that the performance difference is rather small and that it is in the case where the the distance-threshold is at its greatest. Thus, we believe the performance difference is due to the random initialization of the models or the models learning a rough estimate of the keypoints over learning a very fine estimation as this is very difficult when the data is very noisy. We further strengthen our last believe when considering, that the model is the lowest performing model when considering PCK@0.05 but the highest performing model when considering PCK@0.2 and thus generally learns to yield a rough estiamte of the position of the keypoints.
\\
\\
We finally saw in \ref{subsec:pretrain_test_res} how the models for shifting-scalar $s = 1$ had the most difficulty with the wrists and ankles, whereas they performed the best on the ears and shoulders. However, we find this very expected, as the wrists and ankles tend to be the joints that have the highest amount of movements, especially in sports which our pretraining dataset is centered around, and thus are much more difficult to denoise than more static joints such as the ears and shoulders. On the other hand, in the case of using the shifting-scalar $s = 2$ we saw, that the models struggled the most with the elbows and knees, and performed the best on the ears, nose and the ankles. At first glance, one would find this rather unexpecting as the wrists and ankles are no longer the most difficult keypoints. However, we suggest that this could be caused by these keypoints being too close to the heatmap borders, such that when they are shifted a lot towards any of these borders, they will end up at the border, as they cannot exceed the border. Thus, the wrists and ankles are actually not shifted as far as the other keypoints, such as the knees and elbows which would then be the most difficult keypoints, as these keypoints are both shifted a lot and the corresponding joints contain a decent amount of movement.

\subsubsection{Finetuning}
In section \ref{subsec:finetune_train_val_res} we saw how the 3-dimensional convolutional layer generally converged towards the highest validation accuracy, whereas the DeciWatch-based models and the models based on a bidirectional convolutional LSTM tended to plateau in validation accuracy, which even decreased over time for most of these models. This is most likely unexpected to some people, however, if take a look at the evolution of their training loss and validation loss through the finetuning of these models we can see, that the training loss is almost completely monotonically decreasing for all of these models through the whole finetuning-phase, whereas the validation loss is actually increasing. Thus for these models, where the validation loss is actually increasing, we are actually experiencing that they are overfitting. Further, the ClimbAlong-dataset is very small and these models are much bigger than the 3-dimensional convolutional layer (in terms of tuneable parameters), which means that they have a greater likelihood of "remembering" the ClimbAlong-dataset instead of learning its patterns and thus overfitting. We do however find it odd, that there are two Deciwatch-models that are not overfitting - DeciWatch 2.1 and DeciWatch 2.3. We are not too sure why these two models are not overfitting, however, we do find it interesting, that these two models were actually also the two worst performing model during the pretraining-stage.
\\
\\
In section \ref{subsec:finetune_test_res} we saw how the models from the experiments with shifting-scalar $s = 1$ tend to yield better results than the models trained on data with shifting-scalar $s = 2$. We further saw, how the models from experiment 2 generally performs better than the models from experiment 1. Generally however, these performance difference are very minor, which makes sense as the finetuning-data is the same for the two groups. We do note the minor performance differences however, which probably means, that (1) the noise in the finetuning data is more similar to the data with shifting-scalar $s = 1$ than the data with shifting-scalar $s = 2$, and (2) the standard deviation of the peaks of the finetuning data is not changing as much as they do in our experiment 2. 
\\
\\
One could look at Figure \ref{fig:finetune_res} and claim that the pretraining-stage has had no effect, as the validation accuracy starts off very low. However, we disagree with for two reasons. First off, we just noted that the choice of shifting-scalar from the pretraining-stage had an effect on the final results, so the pretraining-stage must have had an effect. Secondly, most of the models actually reach a very high validation accuracy after just a single epoch, which we believe is due to the pretraining. One could argue, that a similar pattern was observed during the pretraining-stage and is thus just a general pattern. However, we have to remember, that the pretraining-dataset is much bigger than the finetuning-dataset and thus just after one epoch in the pretraining-stage, the models have already been trained on a lot of samples, which explains this major performance-jump during the pretrianing-stage.
\\
\\
Lastly, one saw in section \ref{subsec:finetune_test_res} how the models tend to perform the worst on keypoints related to the thumbs, pinkies and index fingers. We see two reasons for this observation. First off, these keypoints were not included in the pretraining dataset, meaning the models have learned to infer the position of these keypoints purely from the finetuning dataset. We believe this reason to only have a minor effect, as if we look at the performance of the models on the keypoints related to the heels and the feet, which were also not included in the pretraining dataset, we see, that the models generally performs only a tiny bit worse on these keypoints, that on the remaining keypoints. The second reason is, that the finger keypoints just have a lot of movement, as the related joints are the joints that are mostly used for bouldering. And as previously stated, keypoints that move a lot are more difficult to predict the position of.

\begin{table}[htbp]
    \begin{tabular}{c||ccc|ccc|ccc}
        \hline
        Accuracy metric & \multicolumn{3}{c}{PCK@0.05} & \multicolumn{3}{c}{PCK@0.1} & \multicolumn{3}{c}{PCK@0.2} \\
        \hline
        Mean threshold distance* & \multicolumn{3}{c}{0.87} & \multicolumn{3}{c}{1.77} & \multicolumn{3}{c}{3.55} \\
        \hline
        Experiment & 1.1 & 1.2 & 1.3 & 1.1 & 1.2 & 1.3 & 1.1 & 1.2 & 1.3 \\
        \hline
        \hline
        Identity function & 21.2 & 21.2 & 21.2 & 65.5 & 65.5 & 65.5 & 84.7 & 84.7 & 84.7 \\
        Conv3D & 58.4 & \textbf{61.4} & \textbf{61.7} & \textbf{98.7} & \textbf{98.9} & \textbf{99.0} & \textbf{99.6} & \textbf{99.8} & \textbf{99.7} \\
        DeciWatch & \textbf{65.1} & 56.0 & 59.6 & 93.5 & 91.1 & 89.5 & 99.1 & 98.9 & 97.6 \\
        bi-ConvLSTM - sum. & 45.7 & 45.0 & 47.6 & 97.3 & 96.9 & 97.0 & \textbf{99.6} & \textbf{99.6} & 99.1 \\
        bi-ConvLSTM - concat. & 44.5 & 46.1 & 48.5 & 97.4 & 97.9 & 97.9 & 99.6 & 99.5 & 99.6 \\
        \hline
    \end{tabular}
    \caption{Testing accuracies of the various developed models for shifting-scalar $k = 1$ on the additional test video. All the accuracies are in percentage. *: The mean maximum distance between the predicted keypoint and corresponding groundtruth keypoint for the prediction to count as being correct, using the units of the heatmap coordinates.}
    \label{tab:finetune_test_accs_3}
\end{table}

\begin{table}[htbp]
    \begin{tabular}{c||ccc|ccc|ccc}
        \hline
        Accuracy metric & \multicolumn{3}{c}{PCK@0.05} & \multicolumn{3}{c}{PCK@0.1} & \multicolumn{3}{c}{PCK@0.2} \\
        \hline
        Mean threshold distance* & \multicolumn{3}{c}{0.87} & \multicolumn{3}{c}{1.77} & \multicolumn{3}{c}{3.55} \\
        \hline
        Experiment & 2.1 & 2.2 & 2.3 & 2.1 & 2.2 & 2.3 & 2.1 & 2.2 & 2.3 \\
        \hline
        \hline
        Identity function & 21.2 & 21.2 & 21.2 & 65.5 & 65.5 & 65.5 & 84.7 & 84.7 & 84.7 \\
        Conv3D & \textbf{56.2} & \textbf{60.0} & \textbf{56.6} & \textbf{98.9} & \textbf{98.8} & \textbf{98.8} & \textbf{99.7} & \textbf{99.7} & \textbf{99.7} \\
        DeciWatch & 35.2 & 48.8 & 40.5 & 75.9 & 89.3 & 79.3 & 94.1 & 98.9 & 95.5 \\
        bi-ConvLSTM - sum. & 44.8 & 46.2 & 45.0 & 96.9 & 95.9 & 97.1 & 99.5 & 99.6 & 99.5 \\
        bi-ConvLSTM - concat. & 45.9 & 47.9 & 46.7 & 96.7 & 97.1 & 98.1 & 99.6 & 99.4 & 99.6 \\
        \hline
    \end{tabular}
    \caption{Testing accuracies of the various developed models for shifting-scalar $k = 2$ on the additional test video. All the accuracies are in percentage. *: The mean maximum distance between the predicted keypoint and corresponding groundtruth keypoint for the prediction to count as being correct, using the units of the heatmap coordinates.}
    \label{tab:finetune_test_accs_4}
\end{table}

\noindent One might argue, that our way of splitting the finetuning dataset into a validation and test set might introduce some bias, as the two sets have overlapping videos and thus videos of the same person that might have some unique movement. To further minimize the likelihood of a bias evaluation of the models, we have in Table \ref{tab:finetune_test_accs_3} and Table \ref{tab:finetune_test_accs_4} tested the models on the longest video of the dataset, which was not used at all in any of the three datasubsets. By comparing these tables against Table \ref{tab:finetune_test_accs_1} and \ref{tab:finetune_test_accs_2} we can clearly see, that the models are not performing any worse than what they did in the evaluation of section \ref{subsec:finetune_test_res}, hence why we would argue, that the evaluation is believable. However, we have to note, that the evaluation in Table \ref{tab:finetune_test_accs_3} and Table \ref{tab:finetune_test_accs_4} does also contain some bias as the evaluation is of a single person who might have some easily predictable movements.

\subsection{Why did the Models Perform Better during Finetuning than during Pretraining?}
All of the finetuned models outperforms their pretrained counterpart. We see a couple of reasons for this. 
\\
\\
First off, the models have been trained om more data, as they have been trained on both the pretraining data and the finetuning data, which of course is always positive.
\\
\\
Secondly, the models have been trained and tested on samples that have very similar to each other. On the other hand, the pretraining dataset consisted of people breakdancing, throwing a baseball, bench pressing and performing sit ups, which are all very different from each other, which could make it hard for the models to learn to generalize. Similarly, the samples from the pretraining dataset were all filmed from very similar camera ankles, whereas the samples from the pretraining data were filmed from all kinds of camera ankles, resulting in the same keypoints being placed at different locations
\\
\\
Further, the BRACE dataset was not fully manually annotated, but instead annotated using a pose estimator and only certain incorrectly predicted poses were manually annotated, where incorrectly predicted poses were detected by a machine learning model. However, if either the machine learning model for detecting incorrectly predicted poses or the pose estimator delivered suboptimal results, then the annotations would also be suboptimal and thus contain some noise, which is very difficult for our models to replicate. The finetuning dataset on the other hand was completely fully annotated by humans, making the annotations much more consistent and thus easier for the models to replicate.
\\
\\
We also see a problem with the Penn Action dataset, which can have an effect on the pretraining results. For the BRACE dataset, it is clearly documented, that the video sequences is filmed using 30 frames per second. For Penn Action, on the other hand, a similar information is not stated anywhere and the video sequences come as individual frames without the duration of the video sequences noted, hence why cannot neither compute the frame rate of these video sequences. This could cause some problems, as two windows of frames, one from each dataset, could span two different durations and thus capture two different duration of context, which can confuse the developed models. On the other hand, all of the video sequences of the finetuning dataset are all filmed using 30 frames per second, making the dataset much more consistent, which again makes the fitting easier.
\\
\\
Lastly, one major reason behind the performance differences behind the models on the pretraining dataset and the finetuning dataset is the perforamnce of the identity functions. If we compare the performance of the identity function on the pretraining dataset in Table \ref{tab:pretrain_test_accs_1} and Table \ref{tab:pretrain_test_accs_2} against the performance of the identity function on the finetuning dataset in Table \ref{tab:finetune_test_accs_1} and Table \ref{tab:finetune_test_accs_2}, we clearly see, that the identity function performs much better during the finetuning than during the pretraining. Thus, a lot less temporal smoothing has to be done by our temporal-inclusive models, making the learning a lot easier.

\subsection{Which Model is the Best for Denoising Human Pose Estimation for Bouldering?}

\subsection{General Reflections}
\begin{itemize}
    \item Hvad gjorde jeg forkert og hvorfor?
\end{itemize}
Generally, we believe that we made the correct choices throughout the execution of the project. However, looking back throughout the project, we do find some bad decision made by us, that could have been avoided or made in another way.

\subsection{Future Work}
If we were to work further with this project, we first find it interesting to experiment with other machine learning methods. We would for instance find it very interesting to test the effects of letting DeciWatch process all frames, instead of only processing every fifth frame. Further, we find it interesting whether or not it would improve the results if DeciWatch was adapted, such that it made use of Vision Transformers as introduced by Dosovitskiy \textit{Et al}. \cite{dosovitskiy2021image}.
\\
\\
Further, we see some potential work in trying to avoid the overfitting during the finetuning-stage that we experienced in section \ref{sec:finetuning} and find it interesting how much this would improve the final results. This could for instance be done by (1) incorporating data augmentation to increase the size and variation of the dataset, for instance by rotation the data or adding some noise, (2) decrease the complexity of the models, such that they have less tuneable parameters, or (3) test various regularization techniques such as weight decay or dropout.
\\
\\
Lastly, we suggest that the models could be retrained multiple times. In section \ref{sec:results} we blaimed some of the results on the random initialization of the weights of the models. By retraining the models multiple times, the likelihood of this randomness being the reasoning for these results decreases and thus we will be sure whether or not our results are due to the randomness of the weights.

\end{document}
\documentclass[./main.tex]{subfiles}

\begin{document}
\section{Introduction}

\subsection{Related Work}
2-dimensional pose estimation can be divided into either being image-based or video-based, where the methods in the latter case use the tempoeral information of the video to perform the pose estimation. 
\\
\\
Image-based methods were initially based on the geometry between the joints of the taget image \cite{6618926, 10.1007/978-3-642-33715-4_19, 6380498}. Following this, were the convolutional-based methods, that used convolutional neural networks \cite{lecun1995convolutional} to perform the pose estimation \cite{https://doi.org/10.48550/arxiv.1602.00134, https://doi.org/10.48550/arxiv.1603.06937, https://doi.org/10.48550/arxiv.1812.08008, https://doi.org/10.48550/arxiv.1703.06870}. More recent methods use transformers \cite{https://doi.org/10.48550/arxiv.1706.03762} to deliver state-of-the-art results \cite{https://doi.org/10.48550/arxiv.2204.12484, https://doi.org/10.48550/arxiv.2012.14214}.
\\
\\
Early video-based methods used 3-dimensional convolutions to capture the temporal information between neighboring frames \cite{https://doi.org/10.48550/arxiv.1506.02897, https://doi.org/10.48550/arxiv.1712.09184}. Other methods use LSTM's \cite{HochSchm97} to capture this temporal information \cite{https://doi.org/10.48550/arxiv.1712.06316, https://doi.org/10.48550/arxiv.2001.08095}. Like in the case of image-based methods, transformers \cite{https://doi.org/10.48550/arxiv.1706.03762} have recently been introduced to the video-based methods to capture the temporal information and deliver state-of-the-art-results \cite{https://doi.org/10.48550/arxiv.2203.08713}.

\subsection{Problem Definition}

\subsection{Reading Guide}

\end{document}
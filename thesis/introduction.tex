\documentclass[./main.tex]{subfiles}

\begin{document}
\section{Introduction}

\begin{itemize}
    \item Imaged-based:
    \begin{itemize}
        \item Geometry between joints in the target image:
        \begin{itemize}
            \item Poselet conditioned pictorial structures
            \item Exploring the spatial hierarchy of mixture models for human pose estimation
            \item  Articulated human detection with flexible mixtures of parts
        \end{itemize}
        \item Convolutional Pose Machine
        \item Stacked Hourglass
        \item OpenPose
        \item HRNet
    \end{itemize}
\end{itemize}

\subsection{Related Work}
2-dimensional pose estimation can be divided into either being image-based or video-based. Image-based methods [\textbf{MANGLER}]... . Video-based methods commonly use the correlating information among the frames of the video to perform the pose estimation. Early video-based methods used 3-dimensional convolutions to capture the correlating information between neighboring frames \cite{https://doi.org/10.48550/arxiv.1506.02897, https://doi.org/10.48550/arxiv.1712.09184}. Other methods use LSTM's \cite{HochSchm97} to capture the correlating information among the frames \cite{https://doi.org/10.48550/arxiv.1712.06316, https://doi.org/10.48550/arxiv.2001.08095}. Recently, transformers \cite{https://doi.org/10.48550/arxiv.1706.03762} have started to being used as a way of capturing the correlating information among the frames \cite{https://doi.org/10.48550/arxiv.2203.08713}.

\subsection{Problem Definition}

\subsection{Reading Guide}

\end{document}
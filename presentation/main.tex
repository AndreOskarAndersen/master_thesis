\documentclass{beamer}
\usepackage{graphicx}
\usepackage{bm}
\usepackage{xcolor}
\usepackage[ruled]{algorithm}
\usepackage{algorithmicx}
\usepackage[noend]{algpseudocode}
\usepackage{amsmath}
\usepackage{amssymb}

\begin{document}
\beamertemplatenavigationsymbolsempty

\title{Temporal Smoothing in 2D Human Pose Estimation for Bouldering}

\author{André Oskar Andersen
\newline \small \texttt{wpr684}}

\institute{Institution of Computer Science, University of Copenhagen}

\date{2023}

\frame{\titlepage}

\begin{frame}
    \frametitle{Introduction}
    \begin{itemize}
        \item<1-3> Increased usage of video analysis in sports.
        \item<2-3> Often requires the position of the players.
        \begin{itemize}
            \item Already developed for popular sports.
            \item Missing for the less popular sports.
        \end{itemize}
        \item<3-3> Problems with the data
        \begin{itemize}
            \item Methods require large quantities
            \item Unusual poses/movements
        \end{itemize}
    \end{itemize}
\end{frame}

\begin{frame}
    \frametitle{Introduction}
    \begin{itemize}
        \item<1-> ClimbAlong at NorthTech ApS
        \begin{itemize}
            \item<1-> Frame-idependent pose-detector for bouldering 
            \item<2-> Proposition: Incorporate temporal information
        \end{itemize}
    \end{itemize}
    % TODO: Tilføj nogle frames her
\end{frame}

\begin{frame}
    \frametitle{Introduction}
    \begin{itemize}
        \item<1-> Aim: extending the ClimbAlong pose-dector to use temporal information.
    \end{itemize}
\end{frame}

\begin{frame}
    \frametitle{The Data}
    \begin{itemize}
        \item<1-> 30 fully annotated videos of climbers with 25 keypoints, totalling to 10,293 frames, provided by ClimbAlong
        \item<2-> Both static and quick movements.
        % TODO: tilføj visualizations
    \end{itemize}
\end{frame}

\begin{frame}
    \frametitle{The Data}
    \begin{itemize}
        \item<1-> Very small dataset
        \item<2-> Instead, pretraing on related datasets and finetune of ClimbAlong
    \end{itemize}
\end{frame}

\begin{frame}
    \frametitle{The Data}
    \begin{itemize}
        \item<1-> The BRACE dataset:
        \begin{itemize}
            \item<1-> Breakdancers
            \item<2-> 1,352 video sequences / 334,538 frames with 17 keypoints.
            \item<3-> Compared to ClimbAlong:
            \begin{itemize}
                \item Swaps between static and quick movements
                \item Less frequent static movements
                \item Quicker movements
            \end{itemize}
        \end{itemize}
    \end{itemize}
    % TODO: tilføj visualizations
\end{frame}

\begin{frame}
    \frametitle{The Data}
    \begin{itemize}
        \item<1-> Penn Action dataset:
        \begin{itemize}
            \item<1-> People performing various actions
            \item<2-> 2,326 video sequences with 13 keypoints and binary visibility-flag.
            \item<3-> Filtered down to 307 video sequences / 26,036 frames.
        \end{itemize}
    \end{itemize}
    % TODO: tilføj visualizations
\end{frame}

\begin{frame}
    \frametitle{The Models}
    \begin{itemize}
        \item Motivation for valg
        \item 3D Conv
        \item DeciWatch
        \item bi-ConvLSTM Model S
        \item bi-ConvLSTM Model C
    \end{itemize}
\end{frame}

\begin{frame}
    \frametitle{The Models}
    \begin{itemize}
        \item<1-> Generally, three approaches
        \begin{enumerate}
            \item 3-dimensional convolutional layer
            \item Convolutional recurrent neural network
            \item Transformer
        \end{enumerate}
        \item<2-> One architecture based on each approach
    \end{itemize}
\end{frame}

\begin{frame}
    \frametitle{The Models}
    \begin{itemize}
        \item<1-> 3DConv
        \begin{itemize}
            \item<1-> 3-dimensional conv. layer + ReLU
            \item<1-> Input/output: heatmaps
            \item<2-> $K \in \mathbb{N}$ filters with $h, w \in \mathbb{N}$ height and width
            \item<2-> $T \in \mathbb{N}$ frames
        \end{itemize}
    \end{itemize}
    % TODO: tilføj visualization
\end{frame}

\begin{frame}
    \frametitle{The Models}
    \begin{itemize}
        \item<1-> bi-ConvLSTM Model S
        \begin{itemize}
            \item<1-> Adoptation of Unipose by Artacho and Savakis
            \item<1-> Bidirectional convolutional LSTM + conv. layers and ReLU
            \item<1-> Processing directions summed together
            \item<2-> Input/output: heatmaps
            \item<2-> $K \in \mathbb{N}$ filters with $h, w \in \mathbb{N}$ height and width
            \item<2-> $T \in \mathbb{N}$ frames
        \end{itemize}
    \end{itemize}
    % TODO: tilføj visualization
\end{frame}

\begin{frame}
    \frametitle{The Models}
    \begin{itemize}
        \item<1-> bi-ConvLSTM Model C
        \begin{itemize}
            \item<1-> Similary to bi-ConvLSTM Model S
            \item<1-> Problem: Prioritization of processing direction
            \item<2-> Solution: Using convolution
        \end{itemize}
    \end{itemize}
    % TODO: tilføj visualization
\end{frame}

\begin{frame}
    \frametitle{The Models}
    \begin{itemize}
        \item<1-> DeciWatch by Zeng \textit{Et al}
        \begin{itemize}
            \item<1-> Transformer-based
            \item<1-> Only considers every $n$th frame
            \item<1-> Processes keypoints
            \item<2-> Encoder: DenoiseNet
            \item<2-> Decoder: RecoverNet
        \end{itemize}
    \end{itemize}
    % TODO: tilføj visualization
\end{frame}

\begin{frame}
    \frametitle{Pretraining}
    \begin{itemize}
        \item<1-> Not training pose-detector 
        \item<2-> Adding noise to input-data
    \end{itemize}
\end{frame}

\begin{frame}
    \frametitle{Pretraining}
    Data preprocessing
    \begin{itemize}
        \item<1-> Discarding images
        \item<2-> Simulating output of pose-detector on ClimbAlong dataset:
        \begin{enumerate}
            \item<2-> Making input data consist of bboxes with side-length of 56 px
            \item<3-> 25 Heatmaps
            \item<4-> Adding noise to input
            \begin{enumerate}
                \item Shifting-scalar
                \item Gaussian filter standard deviation
            \end{enumerate}
        \end{enumerate}
    \end{itemize}
\end{frame}

\begin{frame}
    \frametitle{Pretraining}
    \begin{itemize}
        \item<1-> Data configuration
        \begin{itemize}
            \item $s = 5$ frames
            \item Splitting of data into non-overlapping and repeating subsets
            \item Handling of missing keypoints
        \end{itemize}
    \end{itemize}
\end{frame}

\begin{frame}
    \frametitle{Pretraining}
    \begin{itemize}
        \item<1-> Experiments
        \begin{itemize}
            \item<1-> Sample Gaussian filter standard deviation from $\{1, 1.5, 2, 2.5, 3\}$.
            \item<2-> Fixed standard deviation
            \item<3-> Decreased frame rate
        \end{itemize}
        \item<4-> Two different shifting-scalars $s \in \{1, 2\}$
        \item<5-> DeciWatch - inspected source code:
        \begin{itemize}
            \item Sampling every $5$th frame
        \end{itemize}
    \end{itemize}
\end{frame}

\begin{frame}
    \frametitle{Pretraining}
    \begin{itemize}
        \item Results
    \end{itemize}
\end{frame}

\begin{frame}
    \frametitle{Finetuning}
    \begin{itemize}
        \item Data-preprocessing
        \item Training Details
        \item Results
        \item Visualizations?
        \item Subconclusion
    \end{itemize}
\end{frame}

\begin{frame}
    \frametitle{Finetuning}
    \begin{itemize}
        \item<1-> Freezing pose-detector
        \begin{itemize}
            \item<1-> Quicker fitting
            \item<2-> Greater understanding of results  
        \end{itemize}
    \end{itemize}
\end{frame}

\begin{frame}
    \frametitle{Finetuning}
    \begin{itemize}
        \item<1-> Data preprocessing
        \begin{itemize}
            \item<1-> Making data more similar to pretraining
            \begin{itemize}
                \item Setting negative values to $0$
                \item Normalizing summing up to same value as pretraining heatmaps
            \end{itemize}
            \item<2-> Using predicted bbox to create heatmaps
            \begin{itemize}
                \item<3-> Handling of groundtruth keypoints out of bbox  
            \end{itemize}
        \end{itemize}
    \end{itemize}
\end{frame}

\begin{frame}
    \frametitle{Finetuning}
    \begin{itemize}
        \item<1-> Data configuration
        \begin{itemize}
            \item Careful splitting of dataset
        \end{itemize}
    \end{itemize}
\end{frame}

\begin{frame}
    \frametitle{Finetuning}
    \begin{itemize}
        \item<1-> Results
    \end{itemize}
\end{frame}

\begin{frame}
    \frametitle{Finetuning}
    \begin{itemize}
        \item<1-> Additional experiments
    \end{itemize}
\end{frame}

\begin{frame}
    \frametitle{Finetuning}
    \begin{itemize}
        \item<1-> Additional test results
    \end{itemize}
\end{frame}

\begin{frame}
    \frametitle{Discussion}
\end{frame}

\begin{frame}
    \frametitle{Conclusion}
\end{frame}

\begin{frame}
    \frametitle{Extras: Mistakes Were Made!}
\end{frame}

\end{document}
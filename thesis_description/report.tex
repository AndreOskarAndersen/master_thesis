% This must be in the first 5 lines to tell arXiv to use pdfLaTeX, which is strongly recommended.
\pdfoutput=1
% In particular, the hyperref package requires pdfLaTeX in order to break URLs across lines.

\documentclass[11pt]{article}

% Remove the "review" option to generate the final version.
%\usepackage[]{acl}

% Standard package includes
\usepackage{times}
\usepackage{latexsym}

% For proper rendering and hyphenation of words containing Latin characters (including in bib files)
\usepackage[T1]{fontenc}
% For Vietnamese characters
% \usepackage[T5]{fontenc}
% See https://www.latex-project.org/help/documentation/encguide.pdf for other character sets

% This assumes your files are encoded as UTF8
\usepackage[utf8]{inputenc}

% This is not strictly necessary, and may be commented out,
% but it will improve the layout of the manuscript,
% and will typically save some space.
\usepackage{microtype}

\usepackage{amsmath}
\usepackage{verbatim}
\usepackage{graphicx}
\usepackage{float}
\usepackage{url}


\title{2D Tracking in Rock Climbing with Temporal Smoothing}

\author{André Oskar Andersen \\
  \texttt{wpr684@alumni.ku.dk} \\}

\date{}

\begin{document}

\maketitle

\section{Introduction}
The following paper is the description of my upcomming master thesis in computer science offered at the University of Copenhagen in 2023. It covers the motivation and the most important background behind the thesis, as well as the the aim and learning objectives of the thesis. The thesis will be done in collaboration with ClimbAlong \footnote{\url{https://climbalong.com/}}, which will be described further in Section \ref{sec:collab}.

\section{Background}
Video analysis in sports have throughout the last decade has become more and more common, as these recording contains a lot of important information. By analyzing such a video recording of people engaging in sports, we can for instance help a referre make the correct calls or help the people engaged in the sport develop their technique. However, most of these analyses requires the system to know where the relevant people and objects are in the video recordings. The models that perform such a task have already been developed for the most popular sports, such as football or basketball, and tend to deliver very accurate results. On the other hand, for the less popular sports, such as rock climbing, such models are not as common.
\\
\\
To perform video analysis \textbf{convolutional neural networks} are often used. Convolutional neural networks are a type of neural network that are commonly used for capturing the spatial information of an image or video. These models have a wide range of applications. For finding the people or objects in a video, the most relevant application is the act of \textbf{tracking}, where the model estimates the position of humans or objects in a given video.
\\
\\
However, these convolutional neural networks most oftenly process the frames of an input video independently, leading to suboptimal results. As the individual frames of a video contains data that correlates across the frames, one can make use of \textbf{temporal smoothing} techniques to optimize this performance. Temporal smoothing is a type of technique, that use the information across multiple frames to yield a prediction for a single frame. There are multiple methods to exploit this temporal smoothing. For instance one could use a running average to quickly implement temporal smoothing. Another more sophisticated way of implementing temporal smoothing is by using a variation of a \textbf{recurrent neural network}. Recurrent neural networks are another type of neural network, that are commonly used for capturing the correlating information across a sequence of input data.

\section{Collaboration}
\label{sec:collab}
The thesis will be developed in collaboration with ClimbAlong. ClimbAlong will be providing an annotated dataset of video recordings of people climbing a bouldering wall. This dataset is not a synthetic dataset but instead a real dataset, making the developed results with this dataset be more realistic. Secondly, ClimbAlong will be providing an already developed and trained convolutional neural network for tracking the humans and climbing holds/grips in 2 dimensions in a given video recording. This model does not make use of temporal smoothing, but instead just proces each frame independently of each other. A coorperation agreement with ClimbAlong is currently in development.

\section{Problem Statement}
The aim of this thesis is to incorporate temporal smoothing into a model for 2 dimensional tracking of rock climbers and climbing holds/grips. This may either be done by extending the already 2 dimensional tracking algorithm provided by ClimbAlong or by implementing a whole new model from scratch. Our developed model will thus either be processing features of the input video or be processing the predictions of the ClimbAlong algorithm. We wish to develop such a model by experimenting with various setups and network architectures, such that we can optimize the results. The development of the various setups and network architectures will be based on state-of-the-art methods, either by letting us be inspired by or completely implement these methods.
\\
\\
The development of the experiments will be done by making use of an annotated dataset of video recordings of rock climbers provided by ClimbAlong. This dataset may not be of sufficient size, so we may extend it by either making use of data augmentation or by finding a similar dataset which can be used in addition to the ClimbAlong dataset. 
\\
\\
Once we have developed a sufficient model for 2 dimensional tracking, we will be discussing our approach and the performance of this model. This will include a discussion about what we could have done differently or better, as well as some suggestions for possible future work for our developed model. 


\section{Learning Objectives}
\begin{enumerate}
  \item Present the theory behind convolutional neural networks and recurrent neural networks.
  \item Survey state-of-the-art methods for 2 dimensional tracking with temporal smoothing.
  \item Design, implement and evaluate models for 2 dimensional tracking of climbers and climbing holds/grips with temporal smoothing.
  \item Discuss our approach and the performance of the best-performing developed model.
\end{enumerate}

%\bibliography{custom}

\end{document}